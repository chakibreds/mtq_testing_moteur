\documentclass[12pt,titlepage]{article}

\usepackage{float}
\usepackage[T1]{fontenc}
\usepackage[utf8]{inputenc}
\usepackage[french]{babel} 
\usepackage{amsmath}
\usepackage{amssymb}
\usepackage[top=1.5cm, bottom=1.5cm, left=1.5cm, right=1.5cm]{geometry}
\usepackage{graphicx}
\usepackage{hyperref}

% Bout de code
\usepackage{listings}
\usepackage{color}

\definecolor{mygreen}{rgb}{0,0.6,0}
\definecolor{mygray}{rgb}{0.5,0.5,0.5}
\definecolor{mymauve}{rgb}{0.58,0,0.82}
\definecolor{grey}{rgb}{0.27,0.27,0.27}

\lstset{
  backgroundcolor=\color{white},   % choose the background color; you must add \usepackage{color} or \usepackage{xcolor}; should come as last argument
  basicstyle=\footnotesize,        % the size of the fonts that are used for the code
  breakatwhitespace=false,         % sets if automatic breaks should only happen at whitespace
  breaklines=true,                 % sets automatic line breaking
  captionpos=b,                    % sets the caption-position to bottom
  commentstyle=\color{mygreen},    % comment style
  deletekeywords={...},            % if you want to delete keywords from the given language
  escapeinside={\%*}{*)},          % if you want to add LaTeX within your code
  extendedchars=true,              % lets you use non-ASCII characters; for 8-bits encodings only, does not work with UTF-8
  firstnumber=0,                   % start line enumeration with line 1000
  frame=single,	                   % adds a frame around the code
  keepspaces=true,                 % keeps spaces in text, useful for keeping indentation of code (possibly needs columns=flexible)
  keywordstyle=\color{mygreen},       % keyword style
  %language=C++,                    % the language of the code
  morekeywords={*,...},            % if you want to add more keywords to the set
  numbers=left,                    % where to put the line-numbers; possible values are (none, left, right)
  numbersep=5pt,                   % how far the line-numbers are from the code
  numberstyle=\tiny\color{mygray}, % the style that is used for the line-numbers
  %rulecolor=\color{white},         % if not set, the frame-color may be changed on line-breaks within not-black text (e.g. comments (green here))
  showspaces=false,                % show spaces everywhere adding particular underscores; it overrides 'showstringspaces'
  showstringspaces=false,          % underline spaces within strings only
  showtabs=false,                  % show tabs within strings adding particular underscores
  stepnumber=1,                    % the step between two line-numbers. If it's 1, each line will be numbered
  stringstyle=\color{mymauve},     % string literal style
  tabsize=2,	                   % sets default tabsize to 2 spaces
  literate=
  {á}{{\'a}}1 {é}{{\'e}}1 {í}{{\'i}}1 {ó}{{\'o}}1 {ú}{{\'u}}1
  {Á}{{\'A}}1 {É}{{\'E}}1 {Í}{{\'I}}1 {Ó}{{\'O}}1 {Ú}{{\'U}}1
  {à}{{\`a}}1 {è}{{\`e}}1 {ì}{{\`i}}1 {ò}{{\`o}}1 {ù}{{\`u}}1
  {À}{{\`A}}1 {È}{{\'E}}1 {Ì}{{\`I}}1 {Ò}{{\`O}}1 {Ù}{{\`U}}1
  {ä}{{\"a}}1 {ë}{{\"e}}1 {ï}{{\"i}}1 {ö}{{\"o}}1 {ü}{{\"u}}1
  {Ä}{{\"A}}1 {Ë}{{\"E}}1 {Ï}{{\"I}}1 {Ö}{{\"O}}1 {Ü}{{\"U}}1
  {â}{{\^a}}1 {ê}{{\^e}}1 {î}{{\^i}}1 {ô}{{\^o}}1 {û}{{\^u}}1
  {Â}{{\^A}}1 {Ê}{{\^E}}1 {Î}{{\^I}}1 {Ô}{{\^O}}1 {Û}{{\^U}}1
  {Ã}{{\~A}}1 {ã}{{\~a}}1 {Õ}{{\~O}}1 {õ}{{\~o}}1
  {œ}{{\oe}}1 {Œ}{{\OE}}1 {æ}{{\ae}}1 {Æ}{{\AE}}1 {ß}{{\ss}}1
  {ű}{{\H{u}}}1 {Ű}{{\H{U}}}1 {ő}{{\H{o}}}1 {Ő}{{\H{O}}}1
  {ç}{{\c c}}1 {Ç}{{\c C}}1 {ø}{{\o}}1 {å}{{\r a}}1 {Å}{{\r A}}1
  {€}{{\euro}}1 {£}{{\pounds}}1 {«}{{\guillemotleft}}1
  {»}{{\guillemotright}}1 {ñ}{{\~n}}1 {Ñ}{{\~N}}1 {¿}{{?`}}1
}

\begin{document}

\begin{titlepage}
\newcommand{\HRule}{\rule{\linewidth}{0.5mm}}
\center
\textsc{\LARGE
Université de Montpellier
} \\[1cm]
\begin{figure}[h]
	\begin{minipage}[c]{.46\linewidth}
		\centering
		\includegraphics[width=1\textwidth]{img/fds.png}
	\end{minipage}
	\hfill%
	\begin{minipage}[c]{.46\linewidth}
		\centering
		\includegraphics[width=1\textwidth]{img/univ-montpellier.png}
	\end{minipage}
\end{figure}

\HRule \\[0.4cm]
{ \huge \bfseries Rapport de projet NoSQL \\ Partie II - Évaluation et Analyse des Performances}
\HRule \\[1.5cm]
El Houiti Chakib \\
Kezzoul Massili
\\[1cm]
\today \\ [1cm]
\end{titlepage}

\section*{Introduction}

Ce rapport fait suite à \href{https://github.com/chakibreds/mtq\_moteur\_sparql}{sa première partie} qui concerne l'implémentation d'un moteur de requête \textit{SPARQL} en étoiles en utilisant l'approche \textit{hexastore}. Dans cette partie, nous allons analyser les performances de notre implémentation par rapport à d'autres implémentation. Notamment celle de \href{https://fr.wikipedia.org/wiki/Jena\_(framework)}{Jena}.

Dans un premier temps, nous allons préparer et analyser des bancs d'essais en utilisant \textit{\href{https://dsg.uwaterloo.ca/watdiv/}{WatDiv}\footnote{Waterloo SPARQL Diversity Test Suite.}}. \textit{WatDiv} est système developpé afin de mesurer les performances d'un moteur de requête \textit{SPARQL}. Il consiste en la génération de jeux de données ainsi que des jeux de requêtes.

Dans un second temps, nous allons définir et comparer plusieurs plans de tests afin d'en trouver un (ou plusieurs) qui donne des résultats correctes, significatifs et interprétables.

Enfin, viendra la partie concrète d'évaluation des performances. On executera les plans de tests précédemment réalisés. Suivant les résultats obtenus, nous les présenterons selon des représentations graphiques que nous allons analyser. Nous expliquerons aussi les raisons à la base de ces résultats.

\section{Bancs d'essais}

\subsection{Préparation des bancs d'essais}

\subsection{Analyse des bancs d’essais}

\section{Plan des tests}

\section{Évaluation des performances}

Cette partie est en cours de réalisation.

\section{Conclusion}

\subsection{Apport des évaluations}

\subsection{Perspectives d'amélioration}


\end{document}